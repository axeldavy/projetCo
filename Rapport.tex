\documentclass[a4paper]{article}
\usepackage[latin1]{inputenc}
\usepackage[T1]{fontenc}
\usepackage[francais]{babel}
\usepackage{verbatim}
\usepackage{hyperref}



\author{Axel Davy et Marc Heinrich}
\title{Présentation de notre compilateur}
\begin{document}
\maketitle

\section{Options disponibles}
syntaxe : main (options) filename
\begin{itemize}
     \item {-parse-only : S'arrète après l'analyse syntaxique }
     \item {-type-only : S'arrète après le typage }
\end{itemize}
\section{Difficultés rencontrées}
Outre les erreurs corrigées à la compilation et les erreurs corrigées
automatiquement, on a corrigé pas mal d'oublis lors des tests sur les
fichiers de test.

On a changé légèrement l'arbre de syntaxe abstraite à plusieurs reprises
pour s'adapter à nos besoins.

On a choisi de faire deux arbres différents: un arbre de syntaxe abstraite
pour le parseur (avec un type 'ttype' pour les variables adapté) et un arbre typé avec des
étiquettes supplémentaires pour le typeur (avec un type 'mtype' adapté).

\subsection{Lexeur}
Ici on a eu surtout des erreurs dans le parseur causées par des oublis dans
le lexeur.

On a eu aussi une légère difficulté pour gérer les caractères spéciaux dans
les chaînes de caractères.

\subsection{Parseur}

On a pris quelques libertés par rapport aux règles proposées dans le
dossier au niveau de la représentation des variables et leur
déclarations. En effet nous avons trouvé que la règle <var> n'était pas très pratique pour gérer
les déclarations de variables.

Bien sûr nous avons eu quelques soucis pour régler certains problèmes. On a
eu un peu de mal à gérer les opérateurs unaires, et le
problème du dandling else a été résolu assez tard en donnant une priorité
plus élévée à else. 


\subsection{Typeur}

Pour le typeur, on fait certains choix, comme de rajouter outre les champs
pour le type des expressions, des champs qui pourront être utilisés lors de
la phase de production de code (un champs de localisation pour les
variables locales donnant leur position sur la pile, ou un champ pour
enregistrer la taille des différentes unions et structures déclarées). On a
eu quelque problème avec les structures récursives, qui ne fonctionnait
pas, ou qui après bouclaient quand on essayait de les comparer. 

On a eu du mal aussi à gérer les problèmes de visibilité des variables
locales, et leur éventuelle redéfinition dans des blocs.
Le code du typeur est assez long, notamment à cause des nombreux tests, et
de la conversion d'un type d'arbre à un autre.

Contrairement à gcc qui accepte n'importe quels arguments (en nombre et en type) pour la fonction
main, nous obligeons qu'il y ait soit aucun argument, soit deux arguments
bien typés.

\section{Remarques}
Nous avons pu réaliser que la programmation d'un compilateur est très
fastidieuse. 


Vous pouvez voir notre avancement durant l'année en regardant notre serveur
github: 
\url {https://github.com/axeldavy/projetCo}


\end{document}